% !TeX program = lualatex
% !TeX encoding = UTF-8
%%%%!TEX program = xelatex


\documentclass[]{article}
%\usepackage[UTF8]{ctex}
\usepackage{luatexja-fontspec}
%\usepackage{fontspec}

%使用套件
\usepackage[shortlabels]{enumitem}
\usepackage{graphicx}


%opening
\title{打造個性化的 Neovim 編輯器 }
\author{居正中 Alan Jui}


\begin{document}

% 封面頁
\maketitle
\thispagestyle{empty}
\clearpage

% 摘要
\setcounter{page}{1}
\pagenumbering{roman}

\begin{abstract}
Neovim 0.5 的發行,令 Neovim 推進到一個更加高階的里程碑。對於有心打造符合自己 需求,個人專用編輯器的使用者,Neovim 0.5 支援 LSP 及 Lua Script 的新功能,讓 懷抱著這個夢想的人們,可以有機會美夢成真。\\
\\
 【註】: Neovim Lanage Server protocol (LSP) 主要功能說明,請參考本文件在\ref{sec.lsp.intro}處之內容。\\

當 Neovim 0.5 還在 Beta 版,便已開始使用,如今 Neovim 0.6 都已推出了,個人對 Neovim 的掌握度,仍是一知半解。想要擺脫這狀態,所以需要進行深入的研究;對於 研究該如何開展、如何檢驗是否達標,所以需要設定目標。
\\

最後,將個人在 Django Project 應用的需要,彙整成這個研究的「專案需求」如下:
\begin{itemize}
  \item  程式編碼可透過自動補全,加快輸入及避免打字錯誤
  \item  程式碼在呼叫某 method/function 時,能「顯示用法」,提示該輸入的引數順序, 及應使用的資料類型(Data Type)
  \item  可使用 snippets ,加速編碼工作及避免打字錯誤
  \item  原先在 VS Code 已建置的 Snippets ,能於 Neovim 環境套用
  \item  自動檢查程式碼,確保沒有語法的錯誤
  \item  程式碼已被檢查到的錯誤,可提示:「統計總數」、「標示位置」
  \item  適用於 Django 開發專案
  \item  能依據語法標準(如:autopep8),自動調整程式碼的排版格式
  \item  可編輯及預覽 Markdown 文件,以便可以與 VuePress 整合,作為「技術文件」編輯器
  \item  可使用如 PlantUML 的工具,以文字描述,繪製 UML 圖形(Diagrams)
  \item  可以透過 DAP 與 Neovim 整合,讓 Neovim 可像 VS Code 一樣,當作除錯(Debug)工具來使用
\end{itemize} \\


Neovim 在執行時期,對於設定檔存放目錄及插件存放目錄,有其預設如下:
\begin{itemize}
  \item  設定(Configuration)存放目錄路徑: ~/.config/nvim/
  \item  執行(Runtime)存放目錄路徑: ~/.local/share/nvim/
  \item  插件存放目錄路徑: ~/.local/share/nvim/site/pack/packer/start/
\end{itemize} \\

目前網路各大高手及高高手所分享的 Neovim 設定,幾乎都是遵循上述預設而成。可是, 對於我這種入門新手,個人的期望是:「在鑽研 Neovim 的過程中,需要不斷參考,各個高人們的心得成果,然後自行實作、驗證自己是否理解,最後決定是否要採用,納入本專案的產出: my-nvim 」。所以,我需要兩個相互不影嚮的「工作空間」,一個為參考用;另一個則為實作用。
\\

基於上述的這個需求,my-nvim 被設計成不會佔用 Neovim 的預設目錄路徑: ~/.config/nvim 及 ~/.local/share/nvim 。這個設計,固然能帶來各自獨立的好處; 但也有副作用的麻煩:您得改變以 nvim 指令啟動 Neovim 的習慣。
\end{abstract}
\clearpage

% 目錄
\tableofcontents
\clearpage

% 文件章節開始
\setcounter{page}{1}
\pagenumbering{arabic}

\section{安裝前準備作業(Prerequisites)}

安裝 Neovim 之前,以下四大開發作業環境最好先備妥:
\begin{enumerate}
	\item 作業系統 Shell 環境
	\item Python 開發環境
	\item Node.js 開發環境
	\item Lua 開發環境
\end{enumerate}

\begin{figure}[htbp]
	\centering
	\includegraphics[width=0.7\linewidth]{figures/Neovim-logo}
	\caption{Neovim 0.5 為文字編輯器開創應用新紀元}
	\label{fig:neovim-logo}
\end{figure}



\subsection{完成作業系統 Shell 環境設定}

\subsection{完成 Python 開發環境設定}

\subsection{完成 Node.js 開發環境設定}

\subsection{完成 Lua 開發環境設定}

\clearpage

\section{安裝與設定作業(Setup process)}
\clearpage

\section{後續規劃(Todos)}
\clearpage

\section{快捷鍵(Bindings)}
\clearpage

\section{使用插件(Plugins)}
\clearpage

\section{參考資料(References)}
\clearpage

\subsection{Neovim Language Server Protocol (LSP)}
\label{sec.lsp.intro}

\paragraph{Neovim LSP 主要功能摘要:}

\begin{itemize}
	\item Go to definition:定義跳轉
	\item (auto)completion:自動補全
	\item Code Actions (automatic formatting, organize imports, ...): 程式碼操作,如:自動調整排版、組識 import 順序...等
	\item Show method signatures:顯示用法
	\item Show/go to references:顯示/跳轉引用處
	\item snippets:程式碼片段
\end{itemize}


\end{document}
